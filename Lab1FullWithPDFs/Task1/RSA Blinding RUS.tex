\documentclass[11pt]{article}

    \usepackage[breakable]{tcolorbox}
    \usepackage{parskip} % Stop auto-indenting (to mimic markdown behaviour)
    
    \usepackage{iftex}
    \ifPDFTeX
    	\usepackage[T1]{fontenc}
    	\usepackage{mathpazo}
    \else
    	\usepackage{fontspec}
    \fi
    \usepackage[utf8]{inputenc}
    \usepackage[russian]{babel}
    % Basic figure setup, for now with no caption control since it's done
    % automatically by Pandoc (which extracts ![](path) syntax from Markdown).
    \usepackage{graphicx}
    % Maintain compatibility with old templates. Remove in nbconvert 6.0
    \let\Oldincludegraphics\includegraphics
    % Ensure that by default, figures have no caption (until we provide a
    % proper Figure object with a Caption API and a way to capture that
    % in the conversion process - todo).
    \usepackage{caption}
    \DeclareCaptionFormat{nocaption}{}
    \captionsetup{format=nocaption,aboveskip=0pt,belowskip=0pt}

    \usepackage[Export]{adjustbox} % Used to constrain images to a maximum size
    \adjustboxset{max size={0.9\linewidth}{0.9\paperheight}}
    \usepackage{float}
    \floatplacement{figure}{H} % forces figures to be placed at the correct location
    \usepackage{xcolor} % Allow colors to be defined
    \usepackage{enumerate} % Needed for markdown enumerations to work
    \usepackage{geometry} % Used to adjust the document margins
    \usepackage{amsmath} % Equations
    \usepackage{amssymb} % Equations
    \usepackage{textcomp} % defines textquotesingle
    % Hack from http://tex.stackexchange.com/a/47451/13684:
    \AtBeginDocument{%
        \def\PYZsq{\textquotesingle}% Upright quotes in Pygmentized code
    }
    \usepackage{upquote} % Upright quotes for verbatim code
    \usepackage{eurosym} % defines \euro
    \usepackage[mathletters]{ucs} % Extended unicode (utf-8) support
    \usepackage{fancyvrb} % verbatim replacement that allows latex
    \usepackage{grffile} % extends the file name processing of package graphics 
                         % to support a larger range
    \makeatletter % fix for grffile with XeLaTeX
    \def\Gread@@xetex#1{%
      \IfFileExists{"\Gin@base".bb}%
      {\Gread@eps{\Gin@base.bb}}%
      {\Gread@@xetex@aux#1}%
    }
    \makeatother

    % The hyperref package gives us a pdf with properly built
    % internal navigation ('pdf bookmarks' for the table of contents,
    % internal cross-reference links, web links for URLs, etc.)
    \usepackage{hyperref}
    % The default LaTeX title has an obnoxious amount of whitespace. By default,
    % titling removes some of it. It also provides customization options.
    \usepackage{titling}
    \usepackage{longtable} % longtable support required by pandoc >1.10
    \usepackage{booktabs}  % table support for pandoc > 1.12.2
    \usepackage[inline]{enumitem} % IRkernel/repr support (it uses the enumerate* environment)
    \usepackage[normalem]{ulem} % ulem is needed to support strikethroughs (\sout)
                                % normalem makes italics be italics, not underlines
    \usepackage{mathrsfs}
    

    
    % Colors for the hyperref package
    \definecolor{urlcolor}{rgb}{0,.145,.698}
    \definecolor{linkcolor}{rgb}{.71,0.21,0.01}
    \definecolor{citecolor}{rgb}{.12,.54,.11}

    % ANSI colors
    \definecolor{ansi-black}{HTML}{3E424D}
    \definecolor{ansi-black-intense}{HTML}{282C36}
    \definecolor{ansi-red}{HTML}{E75C58}
    \definecolor{ansi-red-intense}{HTML}{B22B31}
    \definecolor{ansi-green}{HTML}{00A250}
    \definecolor{ansi-green-intense}{HTML}{007427}
    \definecolor{ansi-yellow}{HTML}{DDB62B}
    \definecolor{ansi-yellow-intense}{HTML}{B27D12}
    \definecolor{ansi-blue}{HTML}{208FFB}
    \definecolor{ansi-blue-intense}{HTML}{0065CA}
    \definecolor{ansi-magenta}{HTML}{D160C4}
    \definecolor{ansi-magenta-intense}{HTML}{A03196}
    \definecolor{ansi-cyan}{HTML}{60C6C8}
    \definecolor{ansi-cyan-intense}{HTML}{258F8F}
    \definecolor{ansi-white}{HTML}{C5C1B4}
    \definecolor{ansi-white-intense}{HTML}{A1A6B2}
    \definecolor{ansi-default-inverse-fg}{HTML}{FFFFFF}
    \definecolor{ansi-default-inverse-bg}{HTML}{000000}

    % commands and environments needed by pandoc snippets
    % extracted from the output of `pandoc -s`
    \providecommand{\tightlist}{%
      \setlength{\itemsep}{0pt}\setlength{\parskip}{0pt}}
    \DefineVerbatimEnvironment{Highlighting}{Verbatim}{commandchars=\\\{\}}
    % Add ',fontsize=\small' for more characters per line
    \newenvironment{Shaded}{}{}
    \newcommand{\KeywordTok}[1]{\textcolor[rgb]{0.00,0.44,0.13}{\textbf{{#1}}}}
    \newcommand{\DataTypeTok}[1]{\textcolor[rgb]{0.56,0.13,0.00}{{#1}}}
    \newcommand{\DecValTok}[1]{\textcolor[rgb]{0.25,0.63,0.44}{{#1}}}
    \newcommand{\BaseNTok}[1]{\textcolor[rgb]{0.25,0.63,0.44}{{#1}}}
    \newcommand{\FloatTok}[1]{\textcolor[rgb]{0.25,0.63,0.44}{{#1}}}
    \newcommand{\CharTok}[1]{\textcolor[rgb]{0.25,0.44,0.63}{{#1}}}
    \newcommand{\StringTok}[1]{\textcolor[rgb]{0.25,0.44,0.63}{{#1}}}
    \newcommand{\CommentTok}[1]{\textcolor[rgb]{0.38,0.63,0.69}{\textit{{#1}}}}
    \newcommand{\OtherTok}[1]{\textcolor[rgb]{0.00,0.44,0.13}{{#1}}}
    \newcommand{\AlertTok}[1]{\textcolor[rgb]{1.00,0.00,0.00}{\textbf{{#1}}}}
    \newcommand{\FunctionTok}[1]{\textcolor[rgb]{0.02,0.16,0.49}{{#1}}}
    \newcommand{\RegionMarkerTok}[1]{{#1}}
    \newcommand{\ErrorTok}[1]{\textcolor[rgb]{1.00,0.00,0.00}{\textbf{{#1}}}}
    \newcommand{\NormalTok}[1]{{#1}}
    
    % Additional commands for more recent versions of Pandoc
    \newcommand{\ConstantTok}[1]{\textcolor[rgb]{0.53,0.00,0.00}{{#1}}}
    \newcommand{\SpecialCharTok}[1]{\textcolor[rgb]{0.25,0.44,0.63}{{#1}}}
    \newcommand{\VerbatimStringTok}[1]{\textcolor[rgb]{0.25,0.44,0.63}{{#1}}}
    \newcommand{\SpecialStringTok}[1]{\textcolor[rgb]{0.73,0.40,0.53}{{#1}}}
    \newcommand{\ImportTok}[1]{{#1}}
    \newcommand{\DocumentationTok}[1]{\textcolor[rgb]{0.73,0.13,0.13}{\textit{{#1}}}}
    \newcommand{\AnnotationTok}[1]{\textcolor[rgb]{0.38,0.63,0.69}{\textbf{\textit{{#1}}}}}
    \newcommand{\CommentVarTok}[1]{\textcolor[rgb]{0.38,0.63,0.69}{\textbf{\textit{{#1}}}}}
    \newcommand{\VariableTok}[1]{\textcolor[rgb]{0.10,0.09,0.49}{{#1}}}
    \newcommand{\ControlFlowTok}[1]{\textcolor[rgb]{0.00,0.44,0.13}{\textbf{{#1}}}}
    \newcommand{\OperatorTok}[1]{\textcolor[rgb]{0.40,0.40,0.40}{{#1}}}
    \newcommand{\BuiltInTok}[1]{{#1}}
    \newcommand{\ExtensionTok}[1]{{#1}}
    \newcommand{\PreprocessorTok}[1]{\textcolor[rgb]{0.74,0.48,0.00}{{#1}}}
    \newcommand{\AttributeTok}[1]{\textcolor[rgb]{0.49,0.56,0.16}{{#1}}}
    \newcommand{\InformationTok}[1]{\textcolor[rgb]{0.38,0.63,0.69}{\textbf{\textit{{#1}}}}}
    \newcommand{\WarningTok}[1]{\textcolor[rgb]{0.38,0.63,0.69}{\textbf{\textit{{#1}}}}}
    
    
    % Define a nice break command that doesn't care if a line doesn't already
    % exist.
    \def\br{\hspace*{\fill} \\* }
    % Math Jax compatibility definitions
    \def\gt{>}
    \def\lt{<}
    \let\Oldtex\TeX
    \let\Oldlatex\LaTeX
    \renewcommand{\TeX}{\textrm{\Oldtex}}
    \renewcommand{\LaTeX}{\textrm{\Oldlatex}}
    % Document parameters
    % Document title
    \title{RSA Blinding RUS}
    
    
    
    
    
% Pygments definitions
\makeatletter
\def\PY@reset{\let\PY@it=\relax \let\PY@bf=\relax%
    \let\PY@ul=\relax \let\PY@tc=\relax%
    \let\PY@bc=\relax \let\PY@ff=\relax}
\def\PY@tok#1{\csname PY@tok@#1\endcsname}
\def\PY@toks#1+{\ifx\relax#1\empty\else%
    \PY@tok{#1}\expandafter\PY@toks\fi}
\def\PY@do#1{\PY@bc{\PY@tc{\PY@ul{%
    \PY@it{\PY@bf{\PY@ff{#1}}}}}}}
\def\PY#1#2{\PY@reset\PY@toks#1+\relax+\PY@do{#2}}

\expandafter\def\csname PY@tok@w\endcsname{\def\PY@tc##1{\textcolor[rgb]{0.73,0.73,0.73}{##1}}}
\expandafter\def\csname PY@tok@c\endcsname{\let\PY@it=\textit\def\PY@tc##1{\textcolor[rgb]{0.25,0.50,0.50}{##1}}}
\expandafter\def\csname PY@tok@cp\endcsname{\def\PY@tc##1{\textcolor[rgb]{0.74,0.48,0.00}{##1}}}
\expandafter\def\csname PY@tok@k\endcsname{\let\PY@bf=\textbf\def\PY@tc##1{\textcolor[rgb]{0.00,0.50,0.00}{##1}}}
\expandafter\def\csname PY@tok@kp\endcsname{\def\PY@tc##1{\textcolor[rgb]{0.00,0.50,0.00}{##1}}}
\expandafter\def\csname PY@tok@kt\endcsname{\def\PY@tc##1{\textcolor[rgb]{0.69,0.00,0.25}{##1}}}
\expandafter\def\csname PY@tok@o\endcsname{\def\PY@tc##1{\textcolor[rgb]{0.40,0.40,0.40}{##1}}}
\expandafter\def\csname PY@tok@ow\endcsname{\let\PY@bf=\textbf\def\PY@tc##1{\textcolor[rgb]{0.67,0.13,1.00}{##1}}}
\expandafter\def\csname PY@tok@nb\endcsname{\def\PY@tc##1{\textcolor[rgb]{0.00,0.50,0.00}{##1}}}
\expandafter\def\csname PY@tok@nf\endcsname{\def\PY@tc##1{\textcolor[rgb]{0.00,0.00,1.00}{##1}}}
\expandafter\def\csname PY@tok@nc\endcsname{\let\PY@bf=\textbf\def\PY@tc##1{\textcolor[rgb]{0.00,0.00,1.00}{##1}}}
\expandafter\def\csname PY@tok@nn\endcsname{\let\PY@bf=\textbf\def\PY@tc##1{\textcolor[rgb]{0.00,0.00,1.00}{##1}}}
\expandafter\def\csname PY@tok@ne\endcsname{\let\PY@bf=\textbf\def\PY@tc##1{\textcolor[rgb]{0.82,0.25,0.23}{##1}}}
\expandafter\def\csname PY@tok@nv\endcsname{\def\PY@tc##1{\textcolor[rgb]{0.10,0.09,0.49}{##1}}}
\expandafter\def\csname PY@tok@no\endcsname{\def\PY@tc##1{\textcolor[rgb]{0.53,0.00,0.00}{##1}}}
\expandafter\def\csname PY@tok@nl\endcsname{\def\PY@tc##1{\textcolor[rgb]{0.63,0.63,0.00}{##1}}}
\expandafter\def\csname PY@tok@ni\endcsname{\let\PY@bf=\textbf\def\PY@tc##1{\textcolor[rgb]{0.60,0.60,0.60}{##1}}}
\expandafter\def\csname PY@tok@na\endcsname{\def\PY@tc##1{\textcolor[rgb]{0.49,0.56,0.16}{##1}}}
\expandafter\def\csname PY@tok@nt\endcsname{\let\PY@bf=\textbf\def\PY@tc##1{\textcolor[rgb]{0.00,0.50,0.00}{##1}}}
\expandafter\def\csname PY@tok@nd\endcsname{\def\PY@tc##1{\textcolor[rgb]{0.67,0.13,1.00}{##1}}}
\expandafter\def\csname PY@tok@s\endcsname{\def\PY@tc##1{\textcolor[rgb]{0.73,0.13,0.13}{##1}}}
\expandafter\def\csname PY@tok@sd\endcsname{\let\PY@it=\textit\def\PY@tc##1{\textcolor[rgb]{0.73,0.13,0.13}{##1}}}
\expandafter\def\csname PY@tok@si\endcsname{\let\PY@bf=\textbf\def\PY@tc##1{\textcolor[rgb]{0.73,0.40,0.53}{##1}}}
\expandafter\def\csname PY@tok@se\endcsname{\let\PY@bf=\textbf\def\PY@tc##1{\textcolor[rgb]{0.73,0.40,0.13}{##1}}}
\expandafter\def\csname PY@tok@sr\endcsname{\def\PY@tc##1{\textcolor[rgb]{0.73,0.40,0.53}{##1}}}
\expandafter\def\csname PY@tok@ss\endcsname{\def\PY@tc##1{\textcolor[rgb]{0.10,0.09,0.49}{##1}}}
\expandafter\def\csname PY@tok@sx\endcsname{\def\PY@tc##1{\textcolor[rgb]{0.00,0.50,0.00}{##1}}}
\expandafter\def\csname PY@tok@m\endcsname{\def\PY@tc##1{\textcolor[rgb]{0.40,0.40,0.40}{##1}}}
\expandafter\def\csname PY@tok@gh\endcsname{\let\PY@bf=\textbf\def\PY@tc##1{\textcolor[rgb]{0.00,0.00,0.50}{##1}}}
\expandafter\def\csname PY@tok@gu\endcsname{\let\PY@bf=\textbf\def\PY@tc##1{\textcolor[rgb]{0.50,0.00,0.50}{##1}}}
\expandafter\def\csname PY@tok@gd\endcsname{\def\PY@tc##1{\textcolor[rgb]{0.63,0.00,0.00}{##1}}}
\expandafter\def\csname PY@tok@gi\endcsname{\def\PY@tc##1{\textcolor[rgb]{0.00,0.63,0.00}{##1}}}
\expandafter\def\csname PY@tok@gr\endcsname{\def\PY@tc##1{\textcolor[rgb]{1.00,0.00,0.00}{##1}}}
\expandafter\def\csname PY@tok@ge\endcsname{\let\PY@it=\textit}
\expandafter\def\csname PY@tok@gs\endcsname{\let\PY@bf=\textbf}
\expandafter\def\csname PY@tok@gp\endcsname{\let\PY@bf=\textbf\def\PY@tc##1{\textcolor[rgb]{0.00,0.00,0.50}{##1}}}
\expandafter\def\csname PY@tok@go\endcsname{\def\PY@tc##1{\textcolor[rgb]{0.53,0.53,0.53}{##1}}}
\expandafter\def\csname PY@tok@gt\endcsname{\def\PY@tc##1{\textcolor[rgb]{0.00,0.27,0.87}{##1}}}
\expandafter\def\csname PY@tok@err\endcsname{\def\PY@bc##1{\setlength{\fboxsep}{0pt}\fcolorbox[rgb]{1.00,0.00,0.00}{1,1,1}{\strut ##1}}}
\expandafter\def\csname PY@tok@kc\endcsname{\let\PY@bf=\textbf\def\PY@tc##1{\textcolor[rgb]{0.00,0.50,0.00}{##1}}}
\expandafter\def\csname PY@tok@kd\endcsname{\let\PY@bf=\textbf\def\PY@tc##1{\textcolor[rgb]{0.00,0.50,0.00}{##1}}}
\expandafter\def\csname PY@tok@kn\endcsname{\let\PY@bf=\textbf\def\PY@tc##1{\textcolor[rgb]{0.00,0.50,0.00}{##1}}}
\expandafter\def\csname PY@tok@kr\endcsname{\let\PY@bf=\textbf\def\PY@tc##1{\textcolor[rgb]{0.00,0.50,0.00}{##1}}}
\expandafter\def\csname PY@tok@bp\endcsname{\def\PY@tc##1{\textcolor[rgb]{0.00,0.50,0.00}{##1}}}
\expandafter\def\csname PY@tok@fm\endcsname{\def\PY@tc##1{\textcolor[rgb]{0.00,0.00,1.00}{##1}}}
\expandafter\def\csname PY@tok@vc\endcsname{\def\PY@tc##1{\textcolor[rgb]{0.10,0.09,0.49}{##1}}}
\expandafter\def\csname PY@tok@vg\endcsname{\def\PY@tc##1{\textcolor[rgb]{0.10,0.09,0.49}{##1}}}
\expandafter\def\csname PY@tok@vi\endcsname{\def\PY@tc##1{\textcolor[rgb]{0.10,0.09,0.49}{##1}}}
\expandafter\def\csname PY@tok@vm\endcsname{\def\PY@tc##1{\textcolor[rgb]{0.10,0.09,0.49}{##1}}}
\expandafter\def\csname PY@tok@sa\endcsname{\def\PY@tc##1{\textcolor[rgb]{0.73,0.13,0.13}{##1}}}
\expandafter\def\csname PY@tok@sb\endcsname{\def\PY@tc##1{\textcolor[rgb]{0.73,0.13,0.13}{##1}}}
\expandafter\def\csname PY@tok@sc\endcsname{\def\PY@tc##1{\textcolor[rgb]{0.73,0.13,0.13}{##1}}}
\expandafter\def\csname PY@tok@dl\endcsname{\def\PY@tc##1{\textcolor[rgb]{0.73,0.13,0.13}{##1}}}
\expandafter\def\csname PY@tok@s2\endcsname{\def\PY@tc##1{\textcolor[rgb]{0.73,0.13,0.13}{##1}}}
\expandafter\def\csname PY@tok@sh\endcsname{\def\PY@tc##1{\textcolor[rgb]{0.73,0.13,0.13}{##1}}}
\expandafter\def\csname PY@tok@s1\endcsname{\def\PY@tc##1{\textcolor[rgb]{0.73,0.13,0.13}{##1}}}
\expandafter\def\csname PY@tok@mb\endcsname{\def\PY@tc##1{\textcolor[rgb]{0.40,0.40,0.40}{##1}}}
\expandafter\def\csname PY@tok@mf\endcsname{\def\PY@tc##1{\textcolor[rgb]{0.40,0.40,0.40}{##1}}}
\expandafter\def\csname PY@tok@mh\endcsname{\def\PY@tc##1{\textcolor[rgb]{0.40,0.40,0.40}{##1}}}
\expandafter\def\csname PY@tok@mi\endcsname{\def\PY@tc##1{\textcolor[rgb]{0.40,0.40,0.40}{##1}}}
\expandafter\def\csname PY@tok@il\endcsname{\def\PY@tc##1{\textcolor[rgb]{0.40,0.40,0.40}{##1}}}
\expandafter\def\csname PY@tok@mo\endcsname{\def\PY@tc##1{\textcolor[rgb]{0.40,0.40,0.40}{##1}}}
\expandafter\def\csname PY@tok@ch\endcsname{\let\PY@it=\textit\def\PY@tc##1{\textcolor[rgb]{0.25,0.50,0.50}{##1}}}
\expandafter\def\csname PY@tok@cm\endcsname{\let\PY@it=\textit\def\PY@tc##1{\textcolor[rgb]{0.25,0.50,0.50}{##1}}}
\expandafter\def\csname PY@tok@cpf\endcsname{\let\PY@it=\textit\def\PY@tc##1{\textcolor[rgb]{0.25,0.50,0.50}{##1}}}
\expandafter\def\csname PY@tok@c1\endcsname{\let\PY@it=\textit\def\PY@tc##1{\textcolor[rgb]{0.25,0.50,0.50}{##1}}}
\expandafter\def\csname PY@tok@cs\endcsname{\let\PY@it=\textit\def\PY@tc##1{\textcolor[rgb]{0.25,0.50,0.50}{##1}}}

\def\PYZbs{\char`\\}
\def\PYZus{\char`\_}
\def\PYZob{\char`\{}
\def\PYZcb{\char`\}}
\def\PYZca{\char`\^}
\def\PYZam{\char`\&}
\def\PYZlt{\char`\<}
\def\PYZgt{\char`\>}
\def\PYZsh{\char`\#}
\def\PYZpc{\char`\%}
\def\PYZdl{\char`\$}
\def\PYZhy{\char`\-}
\def\PYZsq{\char`\'}
\def\PYZdq{\char`\"}
\def\PYZti{\char`\~}
% for compatibility with earlier versions
\def\PYZat{@}
\def\PYZlb{[}
\def\PYZrb{]}
\makeatother


    % For linebreaks inside Verbatim environment from package fancyvrb. 
    \makeatletter
        \newbox\Wrappedcontinuationbox 
        \newbox\Wrappedvisiblespacebox 
        \newcommand*\Wrappedvisiblespace {\textcolor{red}{\textvisiblespace}} 
        \newcommand*\Wrappedcontinuationsymbol {\textcolor{red}{\llap{\tiny$\m@th\hookrightarrow$}}} 
        \newcommand*\Wrappedcontinuationindent {3ex } 
        \newcommand*\Wrappedafterbreak {\kern\Wrappedcontinuationindent\copy\Wrappedcontinuationbox} 
        % Take advantage of the already applied Pygments mark-up to insert 
        % potential linebreaks for TeX processing. 
        %        {, <, #, %, $, ' and ": go to next line. 
        %        _, }, ^, &, >, - and ~: stay at end of broken line. 
        % Use of \textquotesingle for straight quote. 
        \newcommand*\Wrappedbreaksatspecials {% 
            \def\PYGZus{\discretionary{\char`\_}{\Wrappedafterbreak}{\char`\_}}% 
            \def\PYGZob{\discretionary{}{\Wrappedafterbreak\char`\{}{\char`\{}}% 
            \def\PYGZcb{\discretionary{\char`\}}{\Wrappedafterbreak}{\char`\}}}% 
            \def\PYGZca{\discretionary{\char`\^}{\Wrappedafterbreak}{\char`\^}}% 
            \def\PYGZam{\discretionary{\char`\&}{\Wrappedafterbreak}{\char`\&}}% 
            \def\PYGZlt{\discretionary{}{\Wrappedafterbreak\char`\<}{\char`\<}}% 
            \def\PYGZgt{\discretionary{\char`\>}{\Wrappedafterbreak}{\char`\>}}% 
            \def\PYGZsh{\discretionary{}{\Wrappedafterbreak\char`\#}{\char`\#}}% 
            \def\PYGZpc{\discretionary{}{\Wrappedafterbreak\char`\%}{\char`\%}}% 
            \def\PYGZdl{\discretionary{}{\Wrappedafterbreak\char`\$}{\char`\$}}% 
            \def\PYGZhy{\discretionary{\char`\-}{\Wrappedafterbreak}{\char`\-}}% 
            \def\PYGZsq{\discretionary{}{\Wrappedafterbreak\textquotesingle}{\textquotesingle}}% 
            \def\PYGZdq{\discretionary{}{\Wrappedafterbreak\char`\"}{\char`\"}}% 
            \def\PYGZti{\discretionary{\char`\~}{\Wrappedafterbreak}{\char`\~}}% 
        } 
        % Some characters . , ; ? ! / are not pygmentized. 
        % This macro makes them "active" and they will insert potential linebreaks 
        \newcommand*\Wrappedbreaksatpunct {% 
            \lccode`\~`\.\lowercase{\def~}{\discretionary{\hbox{\char`\.}}{\Wrappedafterbreak}{\hbox{\char`\.}}}% 
            \lccode`\~`\,\lowercase{\def~}{\discretionary{\hbox{\char`\,}}{\Wrappedafterbreak}{\hbox{\char`\,}}}% 
            \lccode`\~`\;\lowercase{\def~}{\discretionary{\hbox{\char`\;}}{\Wrappedafterbreak}{\hbox{\char`\;}}}% 
            \lccode`\~`\:\lowercase{\def~}{\discretionary{\hbox{\char`\:}}{\Wrappedafterbreak}{\hbox{\char`\:}}}% 
            \lccode`\~`\?\lowercase{\def~}{\discretionary{\hbox{\char`\?}}{\Wrappedafterbreak}{\hbox{\char`\?}}}% 
            \lccode`\~`\!\lowercase{\def~}{\discretionary{\hbox{\char`\!}}{\Wrappedafterbreak}{\hbox{\char`\!}}}% 
            \lccode`\~`\/\lowercase{\def~}{\discretionary{\hbox{\char`\/}}{\Wrappedafterbreak}{\hbox{\char`\/}}}% 
            \catcode`\.\active
            \catcode`\,\active 
            \catcode`\;\active
            \catcode`\:\active
            \catcode`\?\active
            \catcode`\!\active
            \catcode`\/\active 
            \lccode`\~`\~ 	
        }
    \makeatother

    \let\OriginalVerbatim=\Verbatim
    \makeatletter
    \renewcommand{\Verbatim}[1][1]{%
        %\parskip\z@skip
        \sbox\Wrappedcontinuationbox {\Wrappedcontinuationsymbol}%
        \sbox\Wrappedvisiblespacebox {\FV@SetupFont\Wrappedvisiblespace}%
        \def\FancyVerbFormatLine ##1{\hsize\linewidth
            \vtop{\raggedright\hyphenpenalty\z@\exhyphenpenalty\z@
                \doublehyphendemerits\z@\finalhyphendemerits\z@
                \strut ##1\strut}%
        }%
        % If the linebreak is at a space, the latter will be displayed as visible
        % space at end of first line, and a continuation symbol starts next line.
        % Stretch/shrink are however usually zero for typewriter font.
        \def\FV@Space {%
            \nobreak\hskip\z@ plus\fontdimen3\font minus\fontdimen4\font
            \discretionary{\copy\Wrappedvisiblespacebox}{\Wrappedafterbreak}
            {\kern\fontdimen2\font}%
        }%
        
        % Allow breaks at special characters using \PYG... macros.
        \Wrappedbreaksatspecials
        % Breaks at punctuation characters . , ; ? ! and / need catcode=\active 	
        \OriginalVerbatim[#1,codes*=\Wrappedbreaksatpunct]%
    }
    \makeatother

    % Exact colors from NB
    \definecolor{incolor}{HTML}{303F9F}
    \definecolor{outcolor}{HTML}{D84315}
    \definecolor{cellborder}{HTML}{CFCFCF}
    \definecolor{cellbackground}{HTML}{F7F7F7}
    
    % prompt
    \makeatletter
    \newcommand{\boxspacing}{\kern\kvtcb@left@rule\kern\kvtcb@boxsep}
    \makeatother
    \newcommand{\prompt}[4]{
        \ttfamily\llap{{\color{#2}[#3]:\hspace{3pt}#4}}\vspace{-\baselineskip}
    }
    

    
    % Prevent overflowing lines due to hard-to-break entities
    \sloppy 
    % Setup hyperref package
    \hypersetup{
      breaklinks=true,  % so long urls are correctly broken across lines
      colorlinks=true,
      urlcolor=urlcolor,
      linkcolor=linkcolor,
      citecolor=citecolor,
      }
    % Slightly bigger margins than the latex defaults
    
    \geometry{verbose,tmargin=1in,bmargin=1in,lmargin=1in,rmargin=1in}
    
    

\begin{document}
    
    \maketitle
    
    

    
    \section{Атака RSA
Blinding}\label{ux430ux442ux430ux43aux430-rsa-blinding}

\subsection{Введение}\label{ux432ux432ux435ux434ux435ux43dux438ux435}

RSA (названа в честь своих создателей Ronald Linn Rivest, Adi Shamir and
Leonard Adleman) это пример асимметричной криптосистемы, которая может
быть использована для безопасной передачи данных и создания подписей.
рассмотрим базовые принципы работы криптосистемы.

RSA использует мультипликативную группу по модулю \(N=pq\), где \(p\) и
\(q\) - простые числа. Степень мультипликативной группы (по-сути, её
мощность) может быть вычислена при помощи функции Эйлера для составного
числа из двух простых: \(\varphi(N)=(p-1)(q-1)\). Функция считает
количество натуральных чисел меньше \(N\), которые не кратны \(p\) или
\(q\). Поскольку все такие числа взаимно просты с \(N\), они состоят в
мультипликативной группу. Как мы знаем, если возвести любой элемент
конечной мультипликативной группы в степень этой группы, то получим
нейтральный элемент (единицу): \(a^{\varphi(N)}=1, a \in Z^{*}_{N}\).
Поэтому в RSA используют два числа \(e\) (открытая экспонента) и \(d\)
(закрытая экспонента), такие что \(ed=1 mod N\). Пара чисел \((e,N)\)
используется как открытый ключ, а \((d,N)\) как закрытый. Вычисление
\(d\) из открытого ключа является сверхполиномиальной задачей (NP), если
не были сгенерированы слабые \(N\), \(d\) или \(e\). Одним из способов
решения является факторизация \(N\) в произведение \(p\) и \(q\).

Пусть дан открытый текст (число) \(M, M < N\), открытый ключ \((e,N)\) и
закрытый ключ \((d,N)\), шифрование и расшифрование осуществляются
следующим образом:

Шифрование \(C=M^{e}\space mod\space N\)

Расшифрование

\(M=C^{d} \space mod\space N\)

Проверка корректности:

\(C^{d}\space mod\space N=M^{ed}\space mod\space N= M^{ed\space mod\space \varphi(N)}\space mod\space N=M^{1}\space mod\space N= M\space mod\space N\)

\subsection{Preparation}\label{preparation}

Попробуем немного поработать с RSA. Если ещё не установили, установите
Pycryptodome. На Linux и Windows должна сработать следующая команда
(Предварительно надо установить python 3 и pip, но я надеюсь, что вы
справились с этим самостоятельно):

    \begin{tcolorbox}[breakable, size=fbox, boxrule=1pt, pad at break*=1mm,colback=cellbackground, colframe=cellborder]
\prompt{In}{incolor}{1}{\boxspacing}
\begin{Verbatim}[commandchars=\\\{\}]
\PY{o}{!}python3 \PYZhy{}m pip install pycryptodome
\end{Verbatim}
\end{tcolorbox}

    \begin{Verbatim}[commandchars=\\\{\}]
Collecting pycryptodome
  Using cached https://files.pythonhosted.org/packages/54/e4/72132c31a4cedc58848
615502c06cedcce1e1ff703b4c506a7171f005a75/pycryptodome-3.9.6-cp36-cp36m-manylinu
x1\_x86\_64.whl
Installing collected packages: pycryptodome
Successfully installed pycryptodome-3.9.6
    \end{Verbatim}

    После установки надо перезапустить ядро jupyter (круговая стрелка рядом
с "Run"). Если возникнут проблемы, загляните в документацию:
\href{https://pycryptodome.readthedocs.io/en/latest/src/installation.html}{Pycryptodome
installation}.

    \subsection{Примитивная реализация
RSA}\label{ux43fux440ux438ux43cux438ux442ux438ux432ux43dux430ux44f-ux440ux435ux430ux43bux438ux437ux430ux446ux438ux44f-rsa}

Давайте сделаем простейшую версию RSA. Будем использовать открытую
экспоненту \(e=65537\). Обычно используют эту константу, потому что она
переаолняет модуль даже при открытом тексте \(M=2\) и у нее удобное
двоичное представление \(65537_{10}=10000000000000001_{2}\), которое
позволяет эффективно возводить число в степень, используя алгоритм
"Square and multiply".

Сначала сгенерируем \(p\) и \(q\). Функиция getStrongPrime дает
возможность выбрать количество бит в генерируемом простом числе и
проверяет, что \(НОД(p-1,e)=1\)

    \begin{tcolorbox}[breakable, size=fbox, boxrule=1pt, pad at break*=1mm,colback=cellbackground, colframe=cellborder]
\prompt{In}{incolor}{2}{\boxspacing}
\begin{Verbatim}[commandchars=\\\{\}]
\PY{k}{try}\PY{p}{:}
    \PY{k+kn}{from} \PY{n+nn}{Crypto}\PY{n+nn}{.}\PY{n+nn}{Util}\PY{n+nn}{.}\PY{n+nn}{number} \PY{k+kn}{import} \PY{n}{getStrongPrime}\PY{p}{,} \PY{n}{inverse}\PY{p}{,}\PY{n}{bytes\PYZus{}to\PYZus{}long}\PY{p}{,} \PY{n}{long\PYZus{}to\PYZus{}bytes}
\PY{k}{except} \PY{n+ne}{ImportError}\PY{p}{:}
    \PY{n+nb}{print} \PY{p}{(}\PY{l+s+s2}{\PYZdq{}}\PY{l+s+s2}{Pycryptodome not installed}\PY{l+s+s2}{\PYZdq{}}\PY{p}{)}
\end{Verbatim}
\end{tcolorbox}

    \begin{tcolorbox}[breakable, size=fbox, boxrule=1pt, pad at break*=1mm,colback=cellbackground, colframe=cellborder]
\prompt{In}{incolor}{3}{\boxspacing}
\begin{Verbatim}[commandchars=\\\{\}]
\PY{n}{e}\PY{o}{=}\PY{l+m+mi}{65537}
\PY{n}{p}\PY{o}{=}\PY{n}{getStrongPrime}\PY{p}{(}\PY{l+m+mi}{1024}\PY{p}{,}\PY{n}{e}\PY{o}{=}\PY{n}{e}\PY{p}{)}
\PY{n}{q}\PY{o}{=}\PY{n}{getStrongPrime}\PY{p}{(}\PY{l+m+mi}{1024}\PY{p}{,}\PY{n}{e}\PY{o}{=}\PY{n}{e}\PY{p}{)}
\end{Verbatim}
\end{tcolorbox}

    \begin{tcolorbox}[breakable, size=fbox, boxrule=1pt, pad at break*=1mm,colback=cellbackground, colframe=cellborder]
\prompt{In}{incolor}{4}{\boxspacing}
\begin{Verbatim}[commandchars=\\\{\}]
\PY{n}{N}\PY{o}{=}\PY{n}{p}\PY{o}{*}\PY{n}{q}
\PY{n}{phi}\PY{o}{=}\PY{p}{(}\PY{n}{p}\PY{o}{\PYZhy{}}\PY{l+m+mi}{1}\PY{p}{)}\PY{o}{*}\PY{p}{(}\PY{n}{q}\PY{o}{\PYZhy{}}\PY{l+m+mi}{1}\PY{p}{)}
\PY{n}{d}\PY{o}{=}\PY{n}{inverse}\PY{p}{(}\PY{n}{e}\PY{p}{,}\PY{n}{phi}\PY{p}{)}
\PY{n}{public\PYZus{}key}\PY{o}{=}\PY{p}{(}\PY{n}{e}\PY{p}{,}\PY{n}{N}\PY{p}{)}
\PY{n}{private\PYZus{}key}\PY{o}{=}\PY{p}{(}\PY{n}{d}\PY{p}{,}\PY{n}{N}\PY{p}{)}
\end{Verbatim}
\end{tcolorbox}

    Мы успешно сгенерировали ключи, теперь давайте зашифруем сообщение,
расшифруем закрытый текст и проверим, что получили то же самое

    \begin{tcolorbox}[breakable, size=fbox, boxrule=1pt, pad at break*=1mm,colback=cellbackground, colframe=cellborder]
\prompt{In}{incolor}{5}{\boxspacing}
\begin{Verbatim}[commandchars=\\\{\}]
\PY{n}{M}\PY{o}{=}\PY{n}{bytes\PYZus{}to\PYZus{}long}\PY{p}{(}\PY{l+s+sa}{b}\PY{l+s+s1}{\PYZsq{}}\PY{l+s+s1}{Hello, RSA!}\PY{l+s+s1}{\PYZsq{}}\PY{p}{)}
\PY{n}{C}\PY{o}{=}\PY{n+nb}{pow}\PY{p}{(}\PY{n}{M}\PY{p}{,}\PY{n}{e}\PY{p}{,}\PY{n}{N}\PY{p}{)}
\PY{n+nb}{print} \PY{p}{(}\PY{l+s+s1}{\PYZsq{}}\PY{l+s+s1}{C:}\PY{l+s+s1}{\PYZsq{}}\PY{p}{,}\PY{n+nb}{hex}\PY{p}{(}\PY{n}{C}\PY{p}{)}\PY{p}{)}
\PY{n}{M1}\PY{o}{=}\PY{n+nb}{pow}\PY{p}{(}\PY{n}{C}\PY{p}{,}\PY{n}{d}\PY{p}{,}\PY{n}{N}\PY{p}{)}
\PY{k}{assert} \PY{n}{M1}\PY{o}{==}\PY{n}{M}
\PY{n+nb}{print} \PY{p}{(}\PY{l+s+s1}{\PYZsq{}}\PY{l+s+s1}{M1:}\PY{l+s+s1}{\PYZsq{}}\PY{p}{,}\PY{n}{long\PYZus{}to\PYZus{}bytes}\PY{p}{(}\PY{n}{M1}\PY{p}{)}\PY{p}{)}
\end{Verbatim}
\end{tcolorbox}

    \begin{Verbatim}[commandchars=\\\{\}]
C: 0x98f3405c8d1afb28d02379707a07779cfcc727bea6b7b0f0eb6968100ad29489940d4f69dbd
833a550f5c298972033ab7c321cb9e0744961683625d2717b8c426638cd2ec4293ef8b45e7cf675e
c8046149863ae935edb37054f44ed07a24539d4e09bec8eda18264895cd48c28a4ce077c0c24e2f4
95a367c790242ac1f6ccc50ec1984e71fd2fd103d66a7a2239e1dd06f3d5f899c20418a99d3650c9
391e7debf558351b958d0f00087195e3b2ef5c87c1f021480edfd9fc3fd00efd383ebac21a332021
533365be4d65684d6da7c54102da77c2ed4ed43aa8dbf9243621ece4f7f946323c146717b3d61dd6
5d0245c019c96b817010af8926f8aa3bf2e9c
M1: b'Hello, RSA!'
    \end{Verbatim}

    Создание подписи - обратная операция к шифрованию.
\[Sign(M)\equiv Dec(M),\space Check(S) \equiv Enc(S)\] Таким образом
любой, владеющий окрытым ключом, может проверить правильность подписи, а
создать её может только сторона, у которой есть закрытый ключ.
Поздравляю, теперь вы знаете, как шифровать и создавать подаписи при
помощи RSA. Дальше рассмотрим одно из его интересных свойств.

\subsection{RSA Blinding}\label{rsa-blinding}

RSA - это гомоморфное шифрование по отношению к операции умножения.
Отношение является гомоморфизмом групп, если оно операцию одной группы
переводит в опрацию другой группы ,операция и отношение элементов
сохраняются. Если ничего не понятно, не беспокойтесь, я в первый раз,
когда услышал, тоже ничего не понял. Что это значит на практике: пусть у
вас есть два элемента группы \(G_1\) \((x,y)\) и вы применяете к ним
гомоморфное отображение, они будут также связаны в новой группе \(G_2\)
(для RSA \(G_1= G_2\)): \[\varphi(x\cdot y)=\varphi(x)\times\varphi(y)\]
Для шифрования RSA: \[Enc(M_1 \cdot M_2)=Enc(M_1)\times Enc(M_2)=\] То
же самое верно и для расшифрования:
\[Dec(C_1 \times C_2)=Dec(C_1) \cdot Dec(C_2)\] Протестируем это
свойство в python

    \begin{tcolorbox}[breakable, size=fbox, boxrule=1pt, pad at break*=1mm,colback=cellbackground, colframe=cellborder]
\prompt{In}{incolor}{6}{\boxspacing}
\begin{Verbatim}[commandchars=\\\{\}]
\PY{k}{class} \PY{n+nc}{BasicRSA}\PY{p}{:}
    \PY{k}{def} \PY{n+nf+fm}{\PYZus{}\PYZus{}init\PYZus{}\PYZus{}}\PY{p}{(}\PY{n+nb+bp}{self}\PY{p}{,} \PY{n}{e}\PY{p}{,}\PY{n}{p}\PY{p}{,}\PY{n}{q}\PY{p}{)}\PY{p}{:}
        \PY{n+nb+bp}{self}\PY{o}{.}\PY{n}{e}\PY{o}{=}\PY{n}{e}
        \PY{n+nb+bp}{self}\PY{o}{.}\PY{n}{p}\PY{o}{=}\PY{n}{p}
        \PY{n+nb+bp}{self}\PY{o}{.}\PY{n}{q}\PY{o}{=}\PY{n}{q}
        \PY{n+nb+bp}{self}\PY{o}{.}\PY{n}{N}\PY{o}{=}\PY{n}{p}\PY{o}{*}\PY{n}{q}
        \PY{n+nb+bp}{self}\PY{o}{.}\PY{n}{d}\PY{o}{=}\PY{n}{inverse}\PY{p}{(}\PY{n}{e}\PY{p}{,}\PY{p}{(}\PY{n}{p}\PY{o}{\PYZhy{}}\PY{l+m+mi}{1}\PY{p}{)}\PY{o}{*}\PY{p}{(}\PY{n}{q}\PY{o}{\PYZhy{}}\PY{l+m+mi}{1}\PY{p}{)}\PY{p}{)}
    
    \PY{k}{def} \PY{n+nf}{encryptNumber}\PY{p}{(}\PY{n+nb+bp}{self}\PY{p}{,} \PY{n}{m}\PY{p}{)}\PY{p}{:}
        \PY{k}{return} \PY{n+nb}{pow}\PY{p}{(}\PY{n}{m}\PY{p}{,}\PY{n+nb+bp}{self}\PY{o}{.}\PY{n}{e}\PY{p}{,}\PY{n+nb+bp}{self}\PY{o}{.}\PY{n}{N}\PY{p}{)}
    
    \PY{k}{def} \PY{n+nf}{decryptNumber}\PY{p}{(}\PY{n+nb+bp}{self}\PY{p}{,} \PY{n}{c}\PY{p}{)}\PY{p}{:}
        \PY{k}{return} \PY{n+nb}{pow}\PY{p}{(}\PY{n}{c}\PY{p}{,}\PY{n+nb+bp}{self}\PY{o}{.}\PY{n}{d}\PY{p}{,}\PY{n+nb+bp}{self}\PY{o}{.}\PY{n}{N}\PY{p}{)}

\PY{n}{brsa}\PY{o}{=}\PY{n}{BasicRSA}\PY{p}{(}\PY{n}{e}\PY{p}{,}\PY{n}{p}\PY{p}{,}\PY{n}{q}\PY{p}{)} \PY{c+c1}{\PYZsh{}we created these parameters earlier}
\PY{n}{m1}\PY{o}{=}\PY{l+m+mi}{2}
\PY{n}{m2}\PY{o}{=}\PY{l+m+mi}{3}
\PY{n}{m3}\PY{o}{=}\PY{n}{m1}\PY{o}{*}\PY{n}{m2}
\PY{n}{c1}\PY{o}{=}\PY{n}{brsa}\PY{o}{.}\PY{n}{encryptNumber}\PY{p}{(}\PY{n}{m1}\PY{p}{)}
\PY{n}{c2}\PY{o}{=}\PY{n}{brsa}\PY{o}{.}\PY{n}{encryptNumber}\PY{p}{(}\PY{n}{m2}\PY{p}{)}
\PY{n+nb}{print} \PY{p}{(}\PY{l+s+s1}{\PYZsq{}}\PY{l+s+s1}{c1:}\PY{l+s+s1}{\PYZsq{}}\PY{p}{,}\PY{n}{c1}\PY{p}{)}
\PY{n+nb}{print} \PY{p}{(}\PY{l+s+s1}{\PYZsq{}}\PY{l+s+s1}{c2:}\PY{l+s+s1}{\PYZsq{}}\PY{p}{,}\PY{n}{c2}\PY{p}{)}
\PY{n}{c3}\PY{o}{=}\PY{p}{(}\PY{n}{c1}\PY{o}{*}\PY{n}{c2}\PY{p}{)}\PY{o}{\PYZpc{}}\PY{k}{brsa}.N
\PY{n+nb}{print}\PY{p}{(}\PY{l+s+s1}{\PYZsq{}}\PY{l+s+s1}{c3:}\PY{l+s+s1}{\PYZsq{}}\PY{p}{,}\PY{n}{c3}\PY{p}{)}
\PY{n}{m3\PYZus{}dec}\PY{o}{=}\PY{n}{brsa}\PY{o}{.}\PY{n}{decryptNumber}\PY{p}{(}\PY{n}{c3}\PY{p}{)}
\PY{n+nb}{print} \PY{p}{(}\PY{l+s+s1}{\PYZsq{}}\PY{l+s+s1}{m3: }\PY{l+s+si}{\PYZpc{}d}\PY{l+s+s1}{, m3\PYZus{}dec: }\PY{l+s+si}{\PYZpc{}d}\PY{l+s+s1}{\PYZsq{}}\PY{o}{\PYZpc{}}\PY{p}{(}\PY{n}{m3}\PY{p}{,}\PY{n}{m3\PYZus{}dec}\PY{p}{)}\PY{p}{)}
\PY{k}{assert} \PY{n}{m3\PYZus{}dec}\PY{o}{==}\PY{n}{m3}
\end{Verbatim}
\end{tcolorbox}

    \begin{Verbatim}[commandchars=\\\{\}]
c1: 2917785662646598893400674556516703759071653157682606787670108200827863775965
36245310495922376821729112241097566044243314213114199248248618959956992032462919
59971843828577530499391937727849455280466376274023180280811051566420558576090046
39997307133315502987430539399593385090671527799685615736455287812573552757610882
59188708074880376213161888522413138339359729267418233144634478534506833079123369
67403879442593219746216243342414090921955682528477336242701431657678985714164377
88681805020706216540511723275166579436561544209398285547653711536081888046644666
574481152730507745588738692472897383382185663520745893240076
c2: 4699700219532910474099446308513194378286547654354007693316944879517134203166
39033516035489564526647463267862292205374209721423435443115933259608154793907285
14523375168836001256493954554992623899831438449860304545495690875266931303861537
12037958445482674095849068656424903134481885620918278553204292237736030202619489
17098058596891168266246455274537853746415512122754115423879986616138712111657580
81018419379877840156544751827114505281443224757302231619528819233602520309733745
59683051477787335775158266582675969895982815097858644054175038219835190013872610
194040321491590681636260729531079839604284951999974811070504
c3: 1021826351819949017932926548734598860716785415174572433698641026199138812318
07155172334720742043329684009335325838881861120410816121733128168915687720675958
57290002991390827887617402862074354386520097798443917639183793880490158857045378
98406369309911183608293628097017549168107140581420953177048208885129514903707695
24631532472254566587440069169743402218718882227836888276597172352720848804246680
98296525444526159951528867767073349778369049220676022743525760538467511790246995
16950258310316045968748211141458521478652128792519895093962967338044267529997018
4352335358225836742594657124321467482285324705145640991459905
m3: 6, m3\_dec: 6
    \end{Verbatim}

    \subsection{Атакуем
сервер}\label{ux430ux442ux430ux43aux443ux435ux43c-ux441ux435ux440ux432ux435ux440}

Теперь попробуйте применить эти знания к уязвимому серверу. Вы можете
приконнетиться, используя \texttt{nc\ cryptotraining.zone\ 1337} или при
помощи питоновских сокетов.

    \begin{tcolorbox}[breakable, size=fbox, boxrule=1pt, pad at break*=1mm,colback=cellbackground, colframe=cellborder]
\prompt{In}{incolor}{7}{\boxspacing}
\begin{Verbatim}[commandchars=\\\{\}]
\PY{k+kn}{import} \PY{n+nn}{socket}
\PY{k+kn}{import} \PY{n+nn}{re}
\PY{k}{class} \PY{n+nc}{VulnServerClient}\PY{p}{:}
    \PY{k}{def} \PY{n+nf+fm}{\PYZus{}\PYZus{}init\PYZus{}\PYZus{}}\PY{p}{(}\PY{n+nb+bp}{self}\PY{p}{,}\PY{n}{show}\PY{o}{=}\PY{k+kc}{True}\PY{p}{)}\PY{p}{:}
        \PY{l+s+sd}{\PYZdq{}\PYZdq{}\PYZdq{}Ининциализация, подключаемся к серверу\PYZdq{}\PYZdq{}\PYZdq{}}
        \PY{n+nb+bp}{self}\PY{o}{.}\PY{n}{s}\PY{o}{=}\PY{n}{socket}\PY{o}{.}\PY{n}{socket}\PY{p}{(}\PY{n}{socket}\PY{o}{.}\PY{n}{AF\PYZus{}INET}\PY{p}{,}\PY{n}{socket}\PY{o}{.}\PY{n}{SOCK\PYZus{}STREAM}\PY{p}{)}
        \PY{n+nb+bp}{self}\PY{o}{.}\PY{n}{s}\PY{o}{.}\PY{n}{connect}\PY{p}{(}\PY{p}{(}\PY{l+s+s1}{\PYZsq{}}\PY{l+s+s1}{cryptotraining.zone}\PY{l+s+s1}{\PYZsq{}}\PY{p}{,}\PY{l+m+mi}{1337}\PY{p}{)}\PY{p}{)}
        \PY{k}{if} \PY{n}{show}\PY{p}{:}
            \PY{n+nb}{print} \PY{p}{(}\PY{n+nb+bp}{self}\PY{o}{.}\PY{n}{recv\PYZus{}until}\PY{p}{(}\PY{p}{)}\PY{o}{.}\PY{n}{decode}\PY{p}{(}\PY{p}{)}\PY{p}{)}
    \PY{k}{def} \PY{n+nf}{recv\PYZus{}until}\PY{p}{(}\PY{n+nb+bp}{self}\PY{p}{,}\PY{n}{symb}\PY{o}{=}\PY{l+s+sa}{b}\PY{l+s+s1}{\PYZsq{}}\PY{l+s+se}{\PYZbs{}n}\PY{l+s+s1}{\PYZgt{}}\PY{l+s+s1}{\PYZsq{}}\PY{p}{)}\PY{p}{:}
        \PY{l+s+sd}{\PYZdq{}\PYZdq{}\PYZdq{}Получение сообщения с сервера, по умолчанию до приглашения к вводу команды\PYZdq{}\PYZdq{}\PYZdq{}}
        \PY{n}{data}\PY{o}{=}\PY{l+s+sa}{b}\PY{l+s+s1}{\PYZsq{}}\PY{l+s+s1}{\PYZsq{}}
        \PY{k}{while} \PY{k+kc}{True}\PY{p}{:}
            
            \PY{n}{data}\PY{o}{+}\PY{o}{=}\PY{n+nb+bp}{self}\PY{o}{.}\PY{n}{s}\PY{o}{.}\PY{n}{recv}\PY{p}{(}\PY{l+m+mi}{1}\PY{p}{)}
            \PY{k}{if} \PY{n}{data}\PY{p}{[}\PY{o}{\PYZhy{}}\PY{n+nb}{len}\PY{p}{(}\PY{n}{symb}\PY{p}{)}\PY{p}{:}\PY{p}{]}\PY{o}{==}\PY{n}{symb}\PY{p}{:}
                \PY{k}{break}
        \PY{k}{return} \PY{n}{data}
    \PY{k}{def} \PY{n+nf}{get\PYZus{}public\PYZus{}key}\PY{p}{(}\PY{n+nb+bp}{self}\PY{p}{,}\PY{n}{show}\PY{o}{=}\PY{k+kc}{True}\PY{p}{)}\PY{p}{:}
        \PY{l+s+sd}{\PYZdq{}\PYZdq{}\PYZdq{}Получение открытого ключа с сервера\PYZdq{}\PYZdq{}\PYZdq{}}
        \PY{n+nb+bp}{self}\PY{o}{.}\PY{n}{s}\PY{o}{.}\PY{n}{sendall}\PY{p}{(}\PY{l+s+s1}{\PYZsq{}}\PY{l+s+s1}{public}\PY{l+s+se}{\PYZbs{}n}\PY{l+s+s1}{\PYZsq{}}\PY{o}{.}\PY{n}{encode}\PY{p}{(}\PY{p}{)}\PY{p}{)}
        \PY{n}{response}\PY{o}{=}\PY{n+nb+bp}{self}\PY{o}{.}\PY{n}{recv\PYZus{}until}\PY{p}{(}\PY{p}{)}\PY{o}{.}\PY{n}{decode}\PY{p}{(}\PY{p}{)}
        \PY{k}{if} \PY{n}{show}\PY{p}{:}
            \PY{n+nb}{print} \PY{p}{(}\PY{n}{response}\PY{p}{)}
        \PY{n}{e}\PY{o}{=}\PY{n+nb}{int}\PY{p}{(}\PY{n}{re}\PY{o}{.}\PY{n}{search}\PY{p}{(}\PY{l+s+s1}{\PYZsq{}}\PY{l+s+s1}{(?\PYZlt{}=e: )}\PY{l+s+s1}{\PYZbs{}}\PY{l+s+s1}{d+}\PY{l+s+s1}{\PYZsq{}}\PY{p}{,}\PY{n}{response}\PY{p}{)}\PY{o}{.}\PY{n}{group}\PY{p}{(}\PY{l+m+mi}{0}\PY{p}{)}\PY{p}{)}
        \PY{n}{N}\PY{o}{=}\PY{n+nb}{int}\PY{p}{(}\PY{n}{re}\PY{o}{.}\PY{n}{search}\PY{p}{(}\PY{l+s+s1}{\PYZsq{}}\PY{l+s+s1}{(?\PYZlt{}=N: )}\PY{l+s+s1}{\PYZbs{}}\PY{l+s+s1}{d+}\PY{l+s+s1}{\PYZsq{}}\PY{p}{,}\PY{n}{response}\PY{p}{)}\PY{o}{.}\PY{n}{group}\PY{p}{(}\PY{l+m+mi}{0}\PY{p}{)}\PY{p}{)}
        \PY{n+nb+bp}{self}\PY{o}{.}\PY{n}{num\PYZus{}len}\PY{o}{=}\PY{n+nb}{len}\PY{p}{(}\PY{n}{long\PYZus{}to\PYZus{}bytes}\PY{p}{(}\PY{n}{N}\PY{p}{)}\PY{p}{)}
        \PY{k}{return} \PY{p}{(}\PY{n}{e}\PY{p}{,}\PY{n}{N}\PY{p}{)}
    
    \PY{k}{def} \PY{n+nf}{signBytes}\PY{p}{(}\PY{n+nb+bp}{self}\PY{p}{,}\PY{n}{m}\PY{p}{,}\PY{n}{show}\PY{o}{=}\PY{k+kc}{True}\PY{p}{)}\PY{p}{:}
        \PY{l+s+sd}{\PYZdq{}\PYZdq{}\PYZdq{}Получение подписи для выбранного сообщения в байтах с сервера\PYZdq{}\PYZdq{}\PYZdq{}}
        \PY{k}{try}\PY{p}{:}
            \PY{n}{num\PYZus{}len}\PY{o}{=}\PY{n+nb+bp}{self}\PY{o}{.}\PY{n}{num\PYZus{}len}
        \PY{k}{except} \PY{n+ne}{KeyError}\PY{p}{:}
            \PY{n+nb}{print} \PY{p}{(}\PY{l+s+s1}{\PYZsq{}}\PY{l+s+s1}{You need to get the public key from the server first}\PY{l+s+s1}{\PYZsq{}}\PY{p}{)}
            \PY{k}{return}
        \PY{k}{if} \PY{n+nb}{len}\PY{p}{(}\PY{n}{m}\PY{p}{)}\PY{o}{\PYZgt{}}\PY{n}{num\PYZus{}len}\PY{p}{:}
            \PY{n+nb}{print} \PY{p}{(}\PY{l+s+s2}{\PYZdq{}}\PY{l+s+s2}{The message is too long}\PY{l+s+s2}{\PYZdq{}}\PY{p}{)}
            \PY{k}{return}
        \PY{k}{if} \PY{n+nb}{len}\PY{p}{(}\PY{n}{m}\PY{p}{)}\PY{o}{\PYZlt{}}\PY{n}{num\PYZus{}len}\PY{p}{:}
            \PY{n}{m}\PY{o}{=}\PY{n+nb}{bytes}\PY{p}{(}\PY{p}{(}\PY{n}{num\PYZus{}len}\PY{o}{\PYZhy{}}\PY{n+nb}{len}\PY{p}{(}\PY{n}{m}\PY{p}{)}\PY{p}{)}\PY{o}{*}\PY{p}{[}\PY{l+m+mh}{0x0}\PY{p}{]}\PY{p}{)} \PY{o}{+}\PY{n}{m}
        \PY{n}{hex\PYZus{}m}\PY{o}{=}\PY{n}{m}\PY{o}{.}\PY{n}{hex}\PY{p}{(}\PY{p}{)}\PY{o}{.}\PY{n}{encode}\PY{p}{(}\PY{p}{)}
        \PY{n+nb+bp}{self}\PY{o}{.}\PY{n}{s}\PY{o}{.}\PY{n}{sendall}\PY{p}{(}\PY{l+s+sa}{b}\PY{l+s+s1}{\PYZsq{}}\PY{l+s+s1}{sign }\PY{l+s+s1}{\PYZsq{}}\PY{o}{+}\PY{n}{hex\PYZus{}m}\PY{o}{+}\PY{l+s+sa}{b}\PY{l+s+s1}{\PYZsq{}}\PY{l+s+se}{\PYZbs{}n}\PY{l+s+s1}{\PYZsq{}}\PY{p}{)}
        \PY{n}{response}\PY{o}{=}\PY{n+nb+bp}{self}\PY{o}{.}\PY{n}{recv\PYZus{}until}\PY{p}{(}\PY{p}{)}\PY{o}{.}\PY{n}{decode}\PY{p}{(}\PY{p}{)}
        \PY{k}{if} \PY{n}{show}\PY{p}{:}
            \PY{n+nb}{print} \PY{p}{(}\PY{n}{response}\PY{p}{)}
        \PY{k}{if} \PY{n}{response}\PY{o}{.}\PY{n}{find}\PY{p}{(}\PY{l+s+s1}{\PYZsq{}}\PY{l+s+s1}{flag}\PY{l+s+s1}{\PYZsq{}}\PY{p}{)}\PY{o}{!=}\PY{o}{\PYZhy{}}\PY{l+m+mi}{1}\PY{p}{:}
            \PY{n+nb}{print}\PY{p}{(}\PY{l+s+s1}{\PYZsq{}}\PY{l+s+s1}{You tried to submit }\PY{l+s+se}{\PYZbs{}\PYZsq{}}\PY{l+s+s1}{flag}\PY{l+s+se}{\PYZbs{}\PYZsq{}}\PY{l+s+s1}{\PYZsq{}}\PY{p}{)}
            \PY{k}{return} \PY{k+kc}{None}
        \PY{n}{signature\PYZus{}hex}\PY{o}{=}\PY{n}{re}\PY{o}{.}\PY{n}{search}\PY{p}{(}\PY{l+s+s1}{\PYZsq{}}\PY{l+s+s1}{(?\PYZlt{}=Signature: )[0\PYZhy{}9a\PYZhy{}f]+}\PY{l+s+s1}{\PYZsq{}}\PY{p}{,}\PY{n}{response}\PY{p}{)}\PY{o}{.}\PY{n}{group}\PY{p}{(}\PY{l+m+mi}{0}\PY{p}{)}
        \PY{n}{signature\PYZus{}bytes}\PY{o}{=}\PY{n+nb}{bytes}\PY{o}{.}\PY{n}{fromhex}\PY{p}{(}\PY{n}{signature\PYZus{}hex}\PY{p}{)}
        \PY{k}{return} \PY{n}{bytes\PYZus{}to\PYZus{}long}\PY{p}{(}\PY{n}{signature\PYZus{}bytes}\PY{p}{)}
    
    
    \PY{k}{def} \PY{n+nf}{signNumber}\PY{p}{(}\PY{n+nb+bp}{self}\PY{p}{,}\PY{n}{m}\PY{p}{,}\PY{n}{show}\PY{o}{=}\PY{k+kc}{True}\PY{p}{)}\PY{p}{:}
        \PY{l+s+sd}{\PYZdq{}\PYZdq{}\PYZdq{}Получение подписи с сервера для выбранного сообщения в числовом представлении\PYZdq{}\PYZdq{}\PYZdq{}}
        \PY{k}{try}\PY{p}{:}
            \PY{n}{num\PYZus{}len}\PY{o}{=}\PY{n+nb+bp}{self}\PY{o}{.}\PY{n}{num\PYZus{}len}
        \PY{k}{except} \PY{n+ne}{KeyError}\PY{p}{:}
            \PY{n+nb}{print} \PY{p}{(}\PY{l+s+s1}{\PYZsq{}}\PY{l+s+s1}{You need to get the public key from the server first}\PY{l+s+s1}{\PYZsq{}}\PY{p}{)}
            \PY{k}{return}
        \PY{k}{return} \PY{n+nb+bp}{self}\PY{o}{.}\PY{n}{signBytes}\PY{p}{(}\PY{n}{long\PYZus{}to\PYZus{}bytes}\PY{p}{(}\PY{n}{m}\PY{p}{,}\PY{n}{num\PYZus{}len}\PY{p}{)}\PY{p}{,}\PY{n}{show}\PY{p}{)}
        
    \PY{k}{def} \PY{n+nf}{checkSignatureNumber}\PY{p}{(}\PY{n+nb+bp}{self}\PY{p}{,}\PY{n}{c}\PY{p}{,}\PY{n}{show}\PY{o}{=}\PY{k+kc}{True}\PY{p}{)}\PY{p}{:}
        \PY{l+s+sd}{\PYZdq{}\PYZdq{}\PYZdq{}Проверка сигнатуры (на сервере) для подписи в числовом представлении\PYZdq{}\PYZdq{}\PYZdq{}}
        \PY{k}{try}\PY{p}{:}
            \PY{n}{num\PYZus{}len}\PY{o}{=}\PY{n+nb+bp}{self}\PY{o}{.}\PY{n}{num\PYZus{}len}
        \PY{k}{except} \PY{n+ne}{KeyError}\PY{p}{:}
            \PY{n+nb}{print} \PY{p}{(}\PY{l+s+s1}{\PYZsq{}}\PY{l+s+s1}{You need to get the public key from the server first}\PY{l+s+s1}{\PYZsq{}}\PY{p}{)}
            \PY{k}{return}
        \PY{n}{signature\PYZus{}bytes}\PY{o}{=}\PY{n}{long\PYZus{}to\PYZus{}bytes}\PY{p}{(}\PY{n}{c}\PY{p}{,}\PY{n}{num\PYZus{}len}\PY{p}{)}
        \PY{n+nb+bp}{self}\PY{o}{.}\PY{n}{checkSignatureBytes}\PY{p}{(}\PY{n}{signature\PYZus{}bytes}\PY{p}{,}\PY{n}{show}\PY{p}{)}
    
    \PY{k}{def} \PY{n+nf}{checkSignatureBytes}\PY{p}{(}\PY{n+nb+bp}{self}\PY{p}{,}\PY{n}{c}\PY{p}{,}\PY{n}{show}\PY{o}{=}\PY{k+kc}{True}\PY{p}{)}\PY{p}{:}
        \PY{l+s+sd}{\PYZdq{}\PYZdq{}\PYZdq{}Проверка сигнатуры (на сервере) для подписи в байтовом представлении\PYZdq{}\PYZdq{}\PYZdq{}}
        \PY{k}{try}\PY{p}{:}
            \PY{n}{num\PYZus{}len}\PY{o}{=}\PY{n+nb+bp}{self}\PY{o}{.}\PY{n}{num\PYZus{}len}
        \PY{k}{except} \PY{n+ne}{KeyError}\PY{p}{:}
            \PY{n+nb}{print} \PY{p}{(}\PY{l+s+s1}{\PYZsq{}}\PY{l+s+s1}{You need to get the public key from the server first}\PY{l+s+s1}{\PYZsq{}}\PY{p}{)}
            \PY{k}{return}
        \PY{k}{if} \PY{n+nb}{len}\PY{p}{(}\PY{n}{c}\PY{p}{)}\PY{o}{\PYZgt{}}\PY{n}{num\PYZus{}len}\PY{p}{:}
            \PY{n+nb}{print} \PY{p}{(}\PY{l+s+s2}{\PYZdq{}}\PY{l+s+s2}{The message is too long}\PY{l+s+s2}{\PYZdq{}}\PY{p}{)}
            \PY{k}{return}
        
        \PY{n}{hex\PYZus{}c}\PY{o}{=}\PY{n}{c}\PY{o}{.}\PY{n}{hex}\PY{p}{(}\PY{p}{)}\PY{o}{.}\PY{n}{encode}\PY{p}{(}\PY{p}{)}
        \PY{n+nb+bp}{self}\PY{o}{.}\PY{n}{s}\PY{o}{.}\PY{n}{sendall}\PY{p}{(}\PY{l+s+sa}{b}\PY{l+s+s1}{\PYZsq{}}\PY{l+s+s1}{flag }\PY{l+s+s1}{\PYZsq{}}\PY{o}{+}\PY{n}{hex\PYZus{}c}\PY{o}{+}\PY{l+s+sa}{b}\PY{l+s+s1}{\PYZsq{}}\PY{l+s+se}{\PYZbs{}n}\PY{l+s+s1}{\PYZsq{}}\PY{p}{,}\PY{p}{)}
        \PY{n}{response}\PY{o}{=}\PY{n+nb+bp}{self}\PY{o}{.}\PY{n}{recv\PYZus{}until}\PY{p}{(}\PY{l+s+sa}{b}\PY{l+s+s1}{\PYZsq{}}\PY{l+s+se}{\PYZbs{}n}\PY{l+s+s1}{\PYZsq{}}\PY{p}{)}\PY{o}{.}\PY{n}{decode}\PY{p}{(}\PY{p}{)}
        
        \PY{k}{if} \PY{n}{show}\PY{p}{:}
            \PY{n+nb}{print} \PY{p}{(}\PY{n}{response}\PY{p}{)}
        
        \PY{k}{if} \PY{n}{response}\PY{o}{.}\PY{n}{find}\PY{p}{(}\PY{l+s+s1}{\PYZsq{}}\PY{l+s+s1}{Wrong}\PY{l+s+s1}{\PYZsq{}}\PY{p}{)}\PY{o}{!=}\PY{o}{\PYZhy{}}\PY{l+m+mi}{1}\PY{p}{:}
            \PY{n+nb}{print}\PY{p}{(}\PY{l+s+s1}{\PYZsq{}}\PY{l+s+s1}{Wrong signature}\PY{l+s+s1}{\PYZsq{}}\PY{p}{)}
            \PY{n}{x}\PY{o}{=}\PY{n+nb+bp}{self}\PY{o}{.}\PY{n}{recv\PYZus{}until}\PY{p}{(}\PY{p}{)}
            \PY{k}{if} \PY{n}{show}\PY{p}{:}
                \PY{n+nb}{print} \PY{p}{(}\PY{n}{x}\PY{p}{)}
            \PY{k}{return}
        \PY{n}{flag}\PY{o}{=}\PY{n}{re}\PY{o}{.}\PY{n}{search}\PY{p}{(}\PY{l+s+s1}{\PYZsq{}}\PY{l+s+s1}{CRYPTOTRAINING}\PY{l+s+s1}{\PYZbs{}}\PY{l+s+s1}{\PYZob{}}\PY{l+s+s1}{.*}\PY{l+s+s1}{\PYZbs{}}\PY{l+s+s1}{\PYZcb{}}\PY{l+s+s1}{\PYZsq{}}\PY{p}{,}\PY{n}{response}\PY{p}{)}\PY{o}{.}\PY{n}{group}\PY{p}{(}\PY{l+m+mi}{0}\PY{p}{)}
        \PY{n+nb}{print} \PY{p}{(}\PY{l+s+s1}{\PYZsq{}}\PY{l+s+s1}{FLAG: }\PY{l+s+s1}{\PYZsq{}}\PY{p}{,}\PY{n}{flag}\PY{p}{)}
        
    \PY{k}{def} \PY{n+nf+fm}{\PYZus{}\PYZus{}del\PYZus{}\PYZus{}}\PY{p}{(}\PY{n+nb+bp}{self}\PY{p}{)}\PY{p}{:}
        \PY{n+nb+bp}{self}\PY{o}{.}\PY{n}{s}\PY{o}{.}\PY{n}{close}\PY{p}{(}\PY{p}{)}
\end{Verbatim}
\end{tcolorbox}

    \begin{tcolorbox}[breakable, size=fbox, boxrule=1pt, pad at break*=1mm,colback=cellbackground, colframe=cellborder]
\prompt{In}{incolor}{8}{\boxspacing}
\begin{Verbatim}[commandchars=\\\{\}]
\PY{n}{vs}\PY{o}{=}\PY{n}{VulnServerClient}\PY{p}{(}\PY{p}{)}
\PY{p}{(}\PY{n}{e}\PY{p}{,}\PY{n}{N}\PY{p}{)}\PY{o}{=}\PY{n}{vs}\PY{o}{.}\PY{n}{get\PYZus{}public\PYZus{}key}\PY{p}{(}\PY{p}{)}
\end{Verbatim}
\end{tcolorbox}

    \begin{Verbatim}[commandchars=\\\{\}]
Welcome to RSA blinding task
Available commands:
help - print this help
public - show public key
sign <hex(data)> - sign data
flag <hex(signature(b'flag'))> - print flag
quit - quit
>
e: 65537
N: 20159717663186764200842482638329142432479376755681286432561400011207751568770
23937873504239055098886463647821209788938254180637863281345152201173477839435246
47506954302364591564396569321085369361070927857591871209155591733213020275252290
18106368725032056109022369913503577023942696069608771010384365856481001383579432
84411223121576763032862701509742254008778946240450869708632121399086803127321961
48979014368449994422593874530212706423955318848486976509334781242540719122324457
08062597679170291021925633789812405697682134528381868778865376836541179591638312
152472136313757252384761293684336082840137773984575947459061
>
    \end{Verbatim}

    Вы можете подписывать сообщения при помощи методов signNumber (подписать
число) и signBytes (подписать сообщение из байтов)

Проверять подпись можете при помощи методов checkSignatureNumber и
checkSignatureBytes.

Ваша цель - получить правильную подпись для сообщения 'flag'.

Помните, что RSA - это гомоморфизм и решите задание.

Удачи!

    \begin{tcolorbox}[breakable, size=fbox, boxrule=1pt, pad at break*=1mm,colback=cellbackground, colframe=cellborder]
\prompt{In}{incolor}{ }{\boxspacing}
\begin{Verbatim}[commandchars=\\\{\}]

\end{Verbatim}
\end{tcolorbox}


    % Add a bibliography block to the postdoc
    
    
    
\end{document}
